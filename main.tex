\documentclass[12pt, a4paper]{article}
\usepackage{graphicx} 
\usepackage{geometry}
\geometry{a4paper, margin=1in}
\usepackage{tikz}
\usetikzlibrary{calc}
\title{\Huge \textbf{Software Tools And Technology}} 
\date{}
\author{Group 7}

\begin{document}

\begin{figure}
    \centering
    \includegraphics[width=0.3\linewidth]{makaut.jpg}
\end{figure}

\maketitle
\pagenumbering{gobble}:

\begin{tikzpicture}
[remember picture, overlay] \draw[line width = 2pt] ($(current page.north west) + (0.5in, -0.5in)$) rectangle ($(current page.south east) + (-0.5in, 0.5in)$);
\end{tikzpicture}

\begin{center}

\LARGE\textbf{Lab Notebook}

\end{center}

\centering
\vspace{0.5cm}

\bfseries{\underline{Group members:}}

\begin{enumerate}
        \item Jaya Shree Biswas Bsc in IT(DS) (Leader)
        \item Soumyadeep Goswami Bsc in IT(AI)
        \item Suraj Maharaj BCA
        \item Debapriya Dutta Bsc in IT(AI)
        \item Koyena Brahma BCA
\end{enumerate}

\vspace{2cm}
 \textbf{Instructor:} Dr.Ayan Ghosh \\
 \textbf{Course:} Software Tools And Technology

\newpage

\usetikzlibrary{calc}

\begin{tikzpicture}
[remember picture, overlay] \draw[line width = 2pt] ($(current page.north west) + (0.5in, -0.5in)$) rectangle ($(current page.south east) + (-0.5in, 0.5in)$);
\end{tikzpicture}

\centering

\begin{center}
\Large{\bfseries\underline{Lab Notebook Entries}}
\end{center}

\vspace{1.5cm}
\section{Lab Entry by Jaya Shree Biswas}

\vspace{0.7cm}
\subsection{Experiment}

\vspace{0.5cm}

\begin{table}[ht]
\centering
\begin{tabular}{|p{50pt}|p{200pt}|}
\hline
\textbf{Sl. No.} & \textbf{Assignments} \\ \hline
1. & Introduction to Github and Github desktop version installation \\ \hline
\end{tabular}
\end{table}

%\vspace{1.2cm}
\section{Lab Entry by Soumyadeep Goswami}
\subsection{Experiment}

\vspace{0.5cm}
\begin{table}[ht]
\centering
\begin{tabular}{|p{50pt}|p{200pt}|}
\hline
\textbf{Sl. No.} & \textbf{Assignments} \\ \hline
1. & Converting Submit button to Chin Tapak Dum Dum \\ \hline
\end{tabular}
\end{table}

\section{Lab Entry by Suraj Maharaj}
\subsection{Experiment}

\vspace{0.5cm}
\begin{table}[ht]
\centering
\begin{tabular}{|p{50pt}|p{200pt}|}
\hline
\textbf{Sl. No.} & \textbf{Assignments} \\ \hline
1. & Making calculator in C  \\ \hline
\end{tabular}
\end{table}

\newpage

\usetikzlibrary{calc}

\begin{tikzpicture}
[remember picture, overlay] \draw[line width = 2pt] ($(current page.north west) + (0.5in, -0.5in)$) rectangle ($(current page.south east) + (-0.5in, 0.5in)$);
\end{tikzpicture}
\section{Lab Entry by Debapriya Dutta}
\subsection{Experiment}

\vspace{0.5cm}
\begin{table}[ht]
\centering
\begin{tabular}{|p{50pt}|p{200pt}|}
\hline
\textbf{Sl. No.} & \textbf{Assignments} \\ \hline
1. & Creating latex repository on github \\ \hline
\end{tabular}
\end{table}

\vspace{2.5cm}
\section{Lab Entry by Koyena Brahma}
\subsection{Experiment}

\vspace{0.5cm}
\begin{table}[ht]
\centering
\begin{tabular}{|p{50pt}|p{200pt}|}
\hline
\textbf{Sl. No.} & \textbf{Assignments} \\ \hline
1. & Introduction to latex \\ \hline
\end{tabular}
\end{table}

\newpage

\usetikzlibrary{calc}

\begin{tikzpicture}
[remember picture, overlay] \draw[line width = 2pt] ($(current page.north west) + (0.5in, -0.5in)$) rectangle ($(current page.south east) + (-0.5in, 0.5in)$);
\end{tikzpicture}

\section*{\underline{Introduction to GitHub}}
\paragraph{GitHub is a web-based platform for version control using Git, enabling collaboration on software projects. It allows tracking changes, managing code, and working with others seamlessly.
GitHub Desktop is a GUI tool that simplifies Git operations, making it easier for users to manage repositories without using the command line.}

\begin{figure}
    \centering
    \includegraphics[width=0.5\linewidth]{GitHub.png}
\end{figure}
\subsection*{\underline{Installing GitHub Desktop}}
\begin{itemize}
    \item \textbf{Download}: Visit GitHub Desktop and download the version for your OS.
    \item \textbf{Install}: Run the installer and follow the prompts.
    \item \textbf{Sign In}: Open GitHub Desktop and sign in or create a GitHub account.
    \item \textbf{Configure Git}: Set your name and email for commits.
    \item \textbf{Clone/Repository}: Clone existing repositories or create a new one.
    \item \textbf{Commit and Sync}: Make changes, commit them, and push or pull updates from GitHub.
\end{itemize}
\paragraph{GitHub Desktop streamlines Git operations, making version control accessible and straightforward.}


\end{document}
\documentclass{article}

% Packages for additional functionalities
\usepackage{amsmath} % For mathematical symbols and environments
\usepackage{graphicx} % For including images
\usepackage{hyperref} % For clickable hyperlinks
\usepackage{listings} % For including code snippets

% Title and Author
\title{Introduction to \LaTeX}
\author{KOYENA BRAHMA}
\date{\today}

\begin{document}

% Create the title page
\maketitle

% Include logo centered under the title
\begin{figure}[h]
    \centering
    \includegraphics[width=4cm, height=2cm]{LaTeX_logo.svg.png}
   
\end{figure}

% Table of Contents
\tableofcontents
\newpage

\section{Introduction}
\LaTeX{} is a typesetting system that is widely used for producing scientific and mathematical documents due to its powerful handling of formulas and bibliographies. It is also used for other types of documents, from simple letters to complete books.

\section{Basic Document Structure}
A basic \LaTeX{} document has the following structure:

\begin{verbatim}
\documentclass{article}
\begin{document}
% Your content here
\end{document}
\end{verbatim}

\section{Text Formatting}
\LaTeX{} provides various commands for text formatting. Here are some examples:

\begin{itemize}
    \item \textbf{Bold Text} is created using \texttt{\textbackslash textbf\{\}}.
    \item \textit{Italic Text} is created using \texttt{\textbackslash textit\{\}}.
    \item Underlined text can be created using \texttt{\textbackslash underline\{\}}.
\end{itemize}

\section{Mathematical Equations}
One of the most powerful features of \LaTeX{} is its ability to typeset complex mathematical equations. For example:

\begin{equation}
E = mc^2
\end{equation}

Inline equations can be written using the \texttt{\$} symbol, like this: \(a^2 + b^2 = c^2\).

\section{Inserting Images}
You can include images in your \LaTeX{} document using the \texttt{graphicx} package. Here's an example:

\begin{verbatim}
\begin{figure}[h]
    \centering
    \includegraphics[width=0.5\textwidth]{example-image}
    \caption{An example image.}
    \label{fig:example}
\end{figure}
\end{verbatim}

\section{Creating Lists}
\LaTeX{} allows you to create both numbered and bulleted lists easily.

\subsection{Bulleted List}
\begin{itemize}
    \item First item
    \item Second item
    \item Third item
\end{itemize}

\subsection{Numbered List}
\begin{enumerate}
    \item First item
    \item Second item
    \item Third item
\end{enumerate}

\section{Adding Hyperlinks}
You can add hyperlinks in your document using the \texttt{hyperref} package. For example:

\href{https://www.latex-project.org/}{Visit the \LaTeX{} project website.}

\section{Conclusion}
This document provides a brief introduction to some of the basic features of \LaTeX{}. There are many more advanced features that can help you create professional-looking documents.

\end{document}
\documentclass{exam}
\usepackage{graphicx} % Required for inserting images

\begin{document}
	
	\begin{center}
		\fbox{\fbox{\parbox{5.5in}{\centering
					Changing the submit button to Chin Tapak Dum Dum and fixing the disproportionate}}}
	\end{center}
	
	\section{STEPS}
	Changed the submit button to"Chin tapak Dum Dum through this code.
	\begin{figure}[h!]
		\centering
		\includegraphics[width=0.5\linewidth]{JAVA.PNG}
		\caption{JAVA CODE}
		\label{fig:enter-label}
	\end{figure}
	\begin{itemize}
		\item \textbf{Font Size and Style:} The font size and style have been adjusted for improved readability and consistency with the overall design.
		\item \textbf{Background Color:} The background color of the button has been updated to create a more visually appealing and cohesive look.
		\item \textbf{Font Color:} The font color has been modified to ensure strong contrast with the background, enhancing legibility.
		\item \textbf{Element Proportions:} Any disproportionate elements have been corrected to achieve a more balanced and aesthetically pleasing design.
	\end{itemize}
	\section{OUTPUT}
	\begin{figure}
		\centering
		\includegraphics[width=0.5\linewidth]{OUTPUT AFTER.PNG}
		\caption{FINAL OUTPUT}
		\label{fig:enter-label}
	\end{figure}
\end{document}

